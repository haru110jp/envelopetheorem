\documentclass[11pt,a4j,fleqn]{jarticle}
\usepackage{amsmath,amsthm,amssymb}
\usepackage[dvipdfmx]{graphicx}

\title{包絡線定理・レポート}
\author{山岸 敦}
\date{2014/6/7}


\begin{document}

\maketitle

\section{はじめに}

パラメーターを含む関数fについての最適化を考えよう。このとき、そのパラメーターに対して最適化された変数をもとの代入するとパラメーターだけを変数とする関数(価値関数と呼ぶ)ができる。これは経済学の様々な分野で応用されている…のだが、このような言葉の説明だけ読んだところでありがたみが実感できないだろう。ここでは、pythonによって描かれた包絡線定理のグラフをもちいつつ包絡線定理を解説し、その後で描画に用いたpythonプログラムについて解説する。


\section{包絡線定理}

さて、いきなり抽象的な関数に対しての議論は難しいかもしれないので関数系を特定しよう。

変数tとパラメーターxを持つ関数f
\[
f(x, t) =2 t x - t^2
\]

を想定しよう。

数式(番号つき)の例
\begin{equation}
f(x, t)  = -(t - x)^2 + x^2 \label{eq:square}
\end{equation}

数式(番号つき)の例(高さが調整されたカッコ)
\begin{equation}
f(x, t) = -\left(t - \frac{x}{2}\right)^2 + \frac{x^2}{4} \label{eq:square-2}
\end{equation}



数式番号の引用の例:
平方完成の式\eqref{eq:square}より...

\begin{figure}
\begin{center}
\includegraphics{envelope0.pdf}
\end{center}
\caption{1つ目の図の表示}
\label{fig:1}
\end{figure}

\begin{figure}
\begin{center}
\includegraphics{envelope1.pdf}
\end{center}
\caption{2つ目の図の表示}
\label{fig:2}
\end{figure}



引用の例:尾山・安田\cite{OyamaYasuda11}.


\subsection{サブセクションのタイトル}

必要ならサブセクションを作る.



\section{Pythonプログラム}

自分のPythonプログラムの説明を書く.

コードの表示の例
\begin{quote}
\begin{verbatim}
import numpy
from matplotlib import pyplot
x = numpy.arange(0, 10, 0.1)
y = numpy.cos(x)
pyplot.plot(x,y)
pyplot.show()
\end{verbatim}
\end{quote}

\verb|\begin{verbatim}| ... \verb|\end{verbatim}| の中では改行は自動ではされない.
長すぎてページからはみ出す行は自分で適宜改行する.



\begin{thebibliography}{0}
\bibitem{OyamaYasuda11}
尾山大輔・安田洋祐「経済学で出る包絡線定理」『経済セミナー』2011年10・11月号.
\end{thebibliography}

\end{document}